\newpage
\section{G�ltigkeitsbereiche, Namensr�ume und Sichtbarkeit}
	\subsection{Namensr�ume}
		\begin{minipage}[t]{4.5 cm}
			\lstinputlisting{code/namespace.cpp}
		\end{minipage}
		\begin{minipage}[t]{14.5 cm}
			Namensr�ume sind G�ltigkeitsbereiche, in denen beliebige Bezeichner (Variablen, Klassen, Funktionen, andere Namensr�ume, Typen, etc.) deklariert werden k�nnen.
			\begin{compactitem}
				\item Ein Namensraum kann deklariert werden. Alle enthaltenen Objekte werden diesem Namensraum zugeordnet. Auf Bezeichner eines Namensraumes kann mit dem Scope Operator \lc{::} zugegriffen werden.
				\item Einem Namensraum kann ein so genannter Alias zugeordnet werden, �ber den er angesprochen wird. \\
				\lc{namespace FBSSLIB = Financial\_Branch\_and\_System\_Service\_Library;}
				\item Eine so genannte \lc{Using}-Deklaration erlaubt den direkten Zugriff auf einen Bezeichner eines Namensraumes. \\
				\lc{using MyLib1::foo;} \\
				\lc{foo();}
				\item Mit einer so genannten \lc{Using}-Direktive kann auf alle Bezeichner eines Namensraums direkt zugegriffen werden. \\
				\lc{using namespace MyLib1;} \\
				\lc{foo();}			
			\end{compactitem}
		\end{minipage}
		\subsubsection{namenlose Namensr�ume}
			\begin{minipage}[t]{4cm}
				Anstelle von \lc{static} in $C$
			\end{minipage}
			\hspace*{0.5cm}
			\begin{minipage}[t]{3cm}
				\vspace*{-0.2cm}
				\lstinputlisting{code/namelessNamespace.cpp}
			\end{minipage}
			
	\subsection{Deklarationen}
		\subsubsection{Speicherklassenattribute}
			\begin{compactitem}
				\item \lc{auto}: gilt als Standard wenn nichts anderes steht. G�ltigkeitsbereich der \lc{auto} Variablen ist innerhalb des Blockes in dem sie deklariert wurde.
				\item \lc{register}: Hinweis an den Compiler m�glichst die Variable in einem Register abzulegen.
				\item \lc{static}: Variablen leben von ihrer Deklaration bis zum Programmende. Geeignet um zum Bsp. Funktionsaufrufe zu z�hlen anstatt mit globaler Variable.
				\item \lc{extern}: Zugriff auf eine \lc{static} Variable in einem anderen File, welches zu einem gesamten Programm gelinkt wurde.
				\item \lc{mutable}: Klassenelemente mit \lc{const} oder \lc{static} Attributen k�nnen nachtr�glich ver�ndert werden.
			\end{compactitem}
	
		\subsubsection{Typqualifikatoren}
		\begin{compactitem}
			\item \lc{const}: Objekte d�rfen nicht ver�ndert werden. RValues.
			\item \lc{volatile}: Objekte werden evtl. von Aussen im Programmverlauf ver�ndert, und d�rfen daher vom Compiler nicht zu Optimierungszwecken zwischengespeichert werden. Sie werden immer aus dem Hauptspeicher eingelesen.\\
			Verlangsamt das Programm, sollte daher gezielt eingesetzt werden.
		\end{compactitem}
		
		\subsubsection{typedef}
			\begin{minipage}[t]{10 cm}
				\vspace*{-0.3cm}\lstinputlisting[language=C,tabsize=2]{code/typedef.c}
			\end{minipage}
			\begin{minipage}[t]{9 cm}
				\vspace*{-0.35cm}
				Das Schl�sselwort \lc{typedef} erm�glicht die Einf�hrung neuer Bezeichner, die dann im Programm anstelle von anderen Typen verwendet werden k�nnen. \lc{typedef} f�hrt allerdings keine neuen Typen, sondern Synonyme f�r einen existierenden Datentyp ein. Es ist also mehr oder weniger eine Textersetzung.
			\end{minipage}
		 
	\newpage\subsection{Type-cast}
		\subsubsection{implizite-Typumwandlung}
			Ausdr�cke werden bei einer Zuweisung automatisch in den erwarteten Typ umgewandelt.
			\lstinputlisting{code/typecast.c}
	
	\begin{minipage}[t]{4 cm}
		\subsubsection{$C$-Stil}
			\lstinputlisting{code/typecast2.c}
	\end{minipage}
	\hspace*{0.5cm}
	\begin{minipage}[t]{4cm}
		\subsubsection{Funktions-Stil}
			\lstinputlisting{code/typecast3.c}
	\end{minipage}
	\hspace*{0.5cm}
	\begin{minipage}[t]{9.5cm}
		\vspace*{0.1cm}
		{\bfseries Achtung:} Den \lc{C}- und Funktions-Stil sollte in \lc{C++} nicht verwendet werden, da daraus der Grund f�r die Typumwandlung nicht erkannt wird und der Compiler keine Pr�fung durchf�hrt. Deshalb stellt \lc{C++} die folgenden spezifischen und sichereren Typumwandlungen zur Verf�gung.
	\end{minipage}
	
	\subsubsection{Neu in $C++$:}
		\paragraph{\lc{const\_cast}}
			Ausschliesslich bei der Entfernung des \lc{const}-Qualifikators.\\
			Syntax:
			\lstinputlisting{code/constcast.c}
			Beispiel:
			\lstinputlisting{code/constcast_bsp.cpp}
		\paragraph{\lc{static\_cast}}
			Umwandeln eines Klassenobjekt in ein Objekt seiner Basisklasse. Syntax ist analog zu \lc{const\_cast}.\\
			Beispiel:
			\lstinputlisting{code/staticcast.cpp}
		\paragraph{\lc{dynamic\_cast}}	
			Umwandlung von Polymorphen Objekten im Zusammenhang mit dem Typsystem von C++.\\
			Beispiel:
			\lstinputlisting{code/dynamiccast.cpp}
		\paragraph{\lc{reinterpret\_cast}}
			Neue Interpretation der zugrunde liegenden Bitkette.\\
			Beispiel:
			\lstinputlisting{code/reinterpret_cast.cpp}
		
		