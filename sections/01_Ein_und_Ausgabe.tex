\section{Ein- und Ausgabe}
	Um die \lc{C++} Ein- und Ausgaben nutzen zu k�nnen, muss man die Bibliothek iostream einbinden. Das geschieht mit:
	\lstinputlisting{code/iostream.cpp}
	Damit die Ein- und Ausgabebefehle auch wirklich genutzt werden k�nnen, m�ssen sie mithilfe von
	\lstinputlisting{code/using_namespace_std.cpp}
	noch bekanntgegeben werden.\newline	
	\begin{minipage}[t]{11 cm}		
		\subsection{Streamkonzept}
			\begin{compactitem}
				\item Ein \lc{Stream} repr�sentiert einen sequentiellen Datenstrom.
				\item \lc{C++} stellt 4 Standardstreams zur Verf�gung:
					\begin{compactitem}
						\item \lc{cin}: Standard-Eingabesteam
						\item \lc{cout}: Standard-Ausgabestream
						\item \lc{cerr}: Standard-Fehlerausgabestream
						\item \lc{clog}: mit \lc{cerr} gekoppelt
					\end{compactitem}
				\item Alle diese Streams k�nnen auch mit einer Datei verbunden werden.	
				\item Alle Schl�sselw�rter m�ssen immer ganz links auf der Zeile stehen!\newline
			\end{compactitem} 
	\end{minipage}
	\hspace*{0.5cm}	
	\begin{minipage}[t]{7 cm}
		\subsection{Eingabebeispiele}
			\lstinputlisting{code/eingabe_3.cpp}	
	\end{minipage}
	\hspace{0.5cm}
	\subsection{Ausgabebeispiele}
		\lstinputlisting{code/ausgabe_3.cpp}
		\lstinputlisting{code/ausgabe_4.cpp}
		
	\subsection{Formatierte Ein- und Ausgabe: Manipulatoren ohne Parameter}
	\lc{ios}, eine Basisklasse von \lc{iostream}, stellt verschiedene M�glichkeiten (Format Flags) zur Verf�gung, um die Ein- und Ausgabe zu beeinflussen.\newline
		\begin{minipage}[t]{7 cm}
			\subsubsection{boolalpha} 
				\lc{bool}-Werte werden textuell ausgegeben.
				\lstinputlisting{code/boolalpha.cpp}
		\end{minipage}
		\hspace*{0.1cm}	
		\begin{minipage}[t]{2 cm}
				\vspace*{1.2cm}\lstinputlisting{code/boolalpha_aus.cpp}
		\end{minipage}
		\hspace*{0.3cm}	
		\begin{minipage}[t]{7 cm}
			\subsubsection{showbase} 
				Zahlenbasis wird gezeigt.
				\lstinputlisting{code/showbase.cpp}
		\end{minipage}
		\hspace*{0.1cm}	
		\begin{minipage}[t]{2 cm}
				\vspace*{1.2cm}\lstinputlisting{code/showbase_aus.cpp}
		\end{minipage}		
		
		\begin{minipage}[t]{7 cm}
			\subsubsection{showpoint} 
				Dezimalpunkt wird immer ausgegeben.
				\lstinputlisting{code/showpoint.cpp}
		\end{minipage}
		\hspace*{0.1cm}	
		\begin{minipage}[t]{2 cm}
				\vspace*{1.2cm}\lstinputlisting{code/showpoint_aus.cpp}
		\end{minipage}
		\hspace*{0.3cm}	
		\begin{minipage}[t]{7 cm}
			\subsubsection{showpos} 
				Vorzeichen wird bei allen Zahlen angezeigt.
				\lstinputlisting{code/showpos.cpp}
		\end{minipage}
		\hspace*{0.1cm}	
		\begin{minipage}[t]{2 cm}
				\vspace*{1.2cm}\lstinputlisting{code/showpos_aus.cpp}
		\end{minipage}			
		
		\begin{minipage}[t]{7 cm}
			\subsubsection{uppercase} 
				Alles in Grossbuchstaben ausgeben.
				\lstinputlisting{code/uppercase.cpp}
		\end{minipage}
		\hspace*{0.1cm}	
		\begin{minipage}[t]{2 cm}
				\vspace*{1.2cm}\lstinputlisting{code/uppercase_aus.cpp}
		\end{minipage}
		\hspace*{0.3cm}	
		\begin{minipage}[t]{7 cm}
			\subsubsection{dec, hex, oct} 
				Ausgabe erfolgt in dezimal, hexadezimal oder oktal.
				\lstinputlisting[language=C++,tabsize=2]{code/dec_hex_oct.cpp}
		\end{minipage}
		\hspace*{0.1cm}	
		\begin{minipage}[t]{2 cm}
				\vspace*{1.6cm}\lstinputlisting{code/dec_hex_oct_aus.cpp}
		\end{minipage}			
\begin{comment}
		\begin{minipage}[t]{6.5 cm}
			\subsubsection{fixed, scientific} 
				Gleitkommazahlen im Fixpunktformat oder wissenschaftlich.
				\lstinputlisting[language=C++,tabsize=2]{code/fixed_scientific.cpp}
		\end{minipage}
		\hspace*{0.1cm}	
		\begin{minipage}[t]{2.5 cm}
				\vspace*{1.6cm}\lstinputlisting{code/fixed_scientific_aus.cpp}
		\end{minipage}
		\hspace*{0.3cm}	
		\begin{minipage}[t]{7 cm}
			\subsubsection{internal, left, right} 
				Ausgabe innerhalb Feld bzw. links oder rechtsb�ndig.
				\lstinputlisting{code/internal_left_right.cpp} 
		\end{minipage}
		\hspace*{0.1cm}	
		\begin{minipage}[t]{2 cm}
				\vspace*{1.6cm}\lstinputlisting{code/internal_left_right_aus.cpp}
		\end{minipage}	
\end{comment}
	\subsection{Formatierte Ein- und Ausgabe: Manipulatoren mit Parameter} 
		Hier muss zwingend \lc{<iomanip>} eingebunden werden! \\
		\begin{minipage}[t]{7 cm}
			\subsubsection{setw()} 
				Angabe der Feldbreite.
				\lstinputlisting{code/setw.cpp} 
			\subsubsection{setfill()} 
				Auff�llen mit beliebigem F�llzeichen.
				\lstinputlisting{code/setfill.cpp} 
		\end{minipage}
		\hspace*{0.1cm}	
		\begin{minipage}[t]{2 cm}
				\vspace*{1.2cm}\lstinputlisting{code/setw_aus.cpp}
				\vspace*{1cm}\lstinputlisting{code/setfill_aus.cpp}
		\end{minipage}
		\hspace*{0.3cm}	
		\begin{minipage}[t]{7 cm}
			\subsubsection{setprecision()} 
				Angabe der Genauigkeit einer Zahl.
				\lstinputlisting{code/setprecision.cpp} 
		\end{minipage}
		\hspace*{0.1cm}	
		\begin{minipage}[t]{2 cm}
				\vspace*{1.2cm}\lstinputlisting{code/setprecision_aus.cpp}
		\end{minipage}							
	
	\subsection{Streams und Dateien}
		Streams sind Abstraktionen f�r die zeichenweise Ein-/Ausgabe. Dabei ist es egal, ob die Ausgabe auf den Bildschirm erfolgt oder auf eine Datei. Dateien k�nnen also �ber die normalen M�glichkeiten der Ein-/Ausgabe geschrieben bzw. gelesen werden.
	
		\subsubsection{�ffnen von Dateien}
			Das �ffnen einer Datei erfolgt beim Anlegen eines Stream-Objektes beziehungsweise �ber die Funktion \lc{open}. Dabei werden der Name der Datei und der �ffnungsmodus als Parameter �bergeben. \\
			\begin{minipage}[t]{10,4 cm}
				\lstinputlisting{code/file_open.cpp} 
			\end{minipage}
			\hspace*{0.1cm}
			\begin{minipage}[t]{8 cm}
				\begin{tabular}[t]{|p{1.5cm}|p{6cm}|}
					\hline
						\textbf{Modus} & \textbf{Kommentar} \\
					\hline 
						\lc{in} & Datei f�r Eingabe �ffnen \\
						\lc{out} & Datei f�r Ausgabe �ffnen \\
						\lc{app} & Schreiboperationen am Dateiende ausf�hren \\
						\lc{ate} & Nach dem �ffnen der Datei sofort an das Dateiende verzweigen \\
						\lc{trunc} & Zu �ffnende Datei zerst�ren, falls sie bereits existiert \\
						\lc{bin[ary]} & Ein-/Ausgabe wird bin�r durchgef�hrt und nicht im Textmodus \\
					\hline
				\end{tabular}
			\end{minipage}
			
		\subsubsection{Lesen und Setzen von Positionen}
			Die aktuelle Position innerhalb einer Datei kann beliebig ver�ndert werden. Dazu stehen folgende Typen und Methoden zur Verf�gung:
			\begin{compactitem}
				\item \lc{streampos} ist der Typ einer Dateiposition.
				\item \lc{seekg(offset, direction)} zum Beispiel setzt die aktuelle Dateiposition einer Datei. \lc{offset} gibt die Position vom Dateianfang bzw. -ende aus an, \lc{direction} legt fest, von wo aus die Position bestimmt wird.
				\begin{compactitem}
					\item \lc{ios::beg}: Dateianfang
					\item \lc{ios::cur}: Aktuelle Position
					\item \lc{ios::end}: Dateiende
				\end{compactitem}
				\item \lc{tellg} liefert die aktuelle Position in der Datei.
				\item \lc{seekp} und \lc{tellp} sind die entsprechenden Versionen f�r die Put-Varianten.
			\end{compactitem}