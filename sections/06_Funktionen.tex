\newpage
\section{Funktionen}
	Im Vergleich zu \lc{C} funktionieren Funktionen in \lc{C++}	genau gleich, zus�tzlich gibt es aber einige neue, n�tzliche Eigenschaften:
	\begin{compactitem}
		\begin{multicols}{2}
			\item Vorbelegung von Parametern (Default-Argumente)
			\item �berladen von Fuktionen (Overloading)
			\item Operatorfunktionen
			\item inline-Funktionen
		\end{multicols}
	\end{compactitem}	
	\subsection{�berladene Funktionen}
		\begin{minipage}[t]{13cm}
			\begin{compactitem}
				\item Die Identifikation einer Funktion erfolgt �ber die Signatur, nicht nur �ber den Namen. Die Signatur besteht aus dem Namen der Funktion plus der Parameterliste (Reihenfolge, Anzahl, Typ). Der Returntyp wird nicht ber�cksichtigt.
				\item Der Name der Funktionen ist identisch.
				\item Die Implementation muss f�r jede �berladene Funktion separat erfolgen.
				\item Overloading sollte zur�ckhaltend eingesetzt werden. Wenn m�glich sind Default-Argumente vorzuziehen.
			\end{compactitem}
		\end{minipage}
		\hspace*{0.5cm}
		\begin{minipage}[t]{5cm}
			\vspace*{-0.3cm}
			\lstinputlisting{code/overloading.cpp}
		\end{minipage}
	
	\subsection{Vorbelegte Parameter}
		\begin{minipage}[t]{8 cm}
			\begin{compactitem}
				\item Parametern k�nnen im Funktionsprototypen Defaultwerte	zugewiesen werden. 
				\item Beim Funktionsaufruf k�nnen (aber m�ssen nicht) die Parameter mit Defaultwerten weggelassen	werden.
				\item Achtung: Hinter (rechts von) einem Default-Argument darf kein nicht vorbelegter Parameter mehr folgen!
			\end{compactitem}
		\end{minipage}
		\hspace*{0.5cm}
		\begin{minipage}[t]{10 cm}	
			\vspace*{-0.3cm}
			\lstinputlisting[language=C++,tabsize=2]{code/default_parameter.cpp}
		\end{minipage}
					
	\subsection{$inline$-Funktionen vs. $C$-Makros}
		\begin{minipage}[t]{8 cm}
			\subsubsection{$C$-Makros}
				\begin{compactitem}
					\item $C$-Makros werden definiert mit $\#define$.
					\item $C$-Makros bewirken eine reine Textersetzung ohne jegliche Typenpr�fung.
					\item Bei Nebeneffekten (welche zwar vermieden werden sollten) verhalten sich Makros nicht wie beabsichtigt.
					\item $C$-Makros l�sen zwar das Problem mit dem Overhead, sind aber sehr unsicher.				
				\end{compactitem}
				\lstinputlisting[language=C,tabsize=2]{code/c_makro.c}
			\end{minipage}
			\hspace*{0.5cm}
			\begin{minipage}[t]{10 cm}	
				\subsubsection{$inline$-Funktionen}
					\begin{compactitem}
						\item L�sen das Overhead-Problem.
						\item Textersetzung mit Typenpr�fung.
						\item Einsetzen wenn der Codeumfang der Funktion sehr klein ist und die Funktion h�ufig aufgerufen wird (z.B. in Schleifen).
						\item Rekursive Funktionen und Funktionen, auf die mit einem Funktionspointer gezeigt wird, werden nicht ge-$inlined$.
					\end{compactitem}
					\lstinputlisting[language=C++,tabsize=2]{code/inline.cpp}
			\end{minipage}	
		

	
	