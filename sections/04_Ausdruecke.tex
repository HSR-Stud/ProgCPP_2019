\newpage
\section{Ausdr�cke und Operatoren}
	�hnlich wie mathematische Ausdr�cke stellen auch Ausdr�cke in \lc{C++} Berechnungen dar und bestehen aus Operanden und Operatoren. Die Auswertung jedes Ausdrucks liefert einen Wert, der sich aus der Verkn�pfung von Operanden durch Operatoren ergibt. 
	
	\subsection{Auswertungsreihenfolge}
		%Alle Ausdr�cke werden nach bestimmten Regeln ausgewertet. Massgeblich f�r die Art der Auswertung sind dabei Assoziativit�t und Priorit�t der Operatoren. \\
		\vspace*{-0.3cm}\begin{tabular}[t]{|p{1.5cm}|p{4cm}|p{8.5cm}|p{3.5cm}|}
			\hline	
				\textbf{Priorit�t} & \textbf{Operator} & \textbf{Beschreibung} & \textbf{Assoziativit�t} \\
			\hline
				1 & \lc{::} & Bereichsaufl�sung &  von links nach rechts\\
			\hline
				\multirow{5}{*}{2} & \lc{++ -\phantom{}-} & Suffix-/Postfix-Inkrement und -Dekrement & \multirow{5}{*}{von links nach rechts}\\
				& \lc{()} & Funktionsaufruf & \\
				& \lc{[]} & Arrayindizierung & \\
				& \lc{.} & Elementselektion einer Referenz & \\
				& \lc{->} & Elementselektion eines Zeigers & \\
			\hline
				\multirow{9}{*}{3} & \lc{++ -\phantom{}-} & Pr�fix-Inkrement und -Dekrement & \multirow{9}{*}{von rechts nach links} \\
				& \lc{+ -} & un�res plus und un�res Minus & \\
				& \lc{! \~} & logisches NOT und bitweises NOT & \\
				& \lc{(type)} & Typkonvertierung & \\
				& \lc{*} & Dereferenzierung & \\
				& \lc{\&} & Adresse von & \\
				& \lc{sizeof} & Typ-/Objektgr�sse & \\
				& \lc{new, new[]} & Reservierung Dynamischen Speichers & \\
				& \lc{delete, delete[]} & Freigabe Dynamischen Speichers & \\
			\hline
				4 & \lc{.* ->*} & Zeiger-auf-Element & von links nach rechts \\
			\hline
				5 & \lc{* / \%} & Multiplikation, Division und Rest & von links nach rechts \\
			\hline
				6 & \lc{+ -} & Addition und Subtraktion & von links nach rechts \\
			\hline
				7 & \lc{<\phantom{}< >\phantom{}>} & bitweise Rechts- und Linksverschiebung & von links nach rechts \\
			\hline
				\multirow{2}{*}{8} & \lc{< <=} & kleiner-als und kleiner-gleich & \multirow{2}{*}{von links nach rechts}\\
				& \lc{> >=} & gr�sser-als und gr�sser-gleich & \\
			\hline
				9 & \lc{== !=} & gleich und ungleich & von links nach rechts \\
			\hline
				10 & \lc{\&} & bitweises AND & von links nach rechts \\
			\hline
				11 & \lc{\textasciicircum} & bitweise XOR & von links nach rechts \\
			\hline
				12 & \lc{|} & bitweises OR & von links nach rechts \\
			\hline
				13 & \lc{\&\&} & logisches AND & von links nach rechts \\
			\hline
				14 & \lc{||} & logisches OR & von links nach rechts \\
			\hline
				\multirow{6}{*}{15} & \lc{?:} & bedingte Zuweisung & \multirow{6}{*}{von rechts nach links}\\
				& \lc{=} & einfache Zuweisung & \\
				& \lc{+= -=} & Zuweisung nach Addition/Subtraktion & \\
				& \lc{*= /= \%=} & Zuweisung nach Multiplikation, Division, Rest & \\
				& \lc{<\phantom{}<= >\phantom{}>=} & Zuweisung nach Links-, Rechtsverschiebung & \\
				& \lc{\&= \textasciicircum= |=} & Zuweisung nach bitweisem AND, XOR und OR & \\
			\hline
				16 & \lc{throw} & Ausnahme werfen & von rechts nach links \\
			\hline
				17 & \lc{,} & Komma (Sequenzoperator) & von links nach rechts \\
			\hline
		\end{tabular}

		\ \\ (Priorit�t 1 hat Vorrang vor allen anderen)
		
		\subsubsection{Assoziativit�t}
			Die Assoziativit�t gibt Auskunft �ber die Auswertungsreihenfolge der Operanden eines Ausdrucks. So wird zum Beispiel im Ausdruck \lc{p++} zuerst \lc{p} ausgewertet und dann die linke Seite des Operators \lc{++} (\lc{p}) erh�ht, w�hrend der Ausdruck \lc{++p} zuerst \lc{p} erh�ht und dann den Ausdruck auswertet.
			\lstinputlisting[language=C++,tabsize=2]{code/prio1.cpp}
			
		\subsubsection{Priorit�t}
			Die Priorit�t von Operatoren wiederum gibt an, in welcher Reihenfolge die verschiedenen Operanden eines Ausdrucks ausgewertet werden. Die multiplikativen Operatoren weisen zum Beispiel eine h�here Priorit�t als die additiven Operatoren auf.			
